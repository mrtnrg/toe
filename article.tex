\documentclass[]{article}
\usepackage{amsmath}
\setlength{\parindent}{0pt} % no indent at beginning of new paragraph
\setlength{\parskip}{1pt} % insert 1pt high line at beginning of new 
%opening
\title{}
\author{}

\begin{document}
%\maketitle
%\begin{abstract}
%\end{abstract}

\subsection{Event}
\begin{equation}
\underline{x}_i
=
\begin{pmatrix}
	\underline{x}_0\\
	\underline{x}_1\\
	\underline{x}_2\\
	\underline{x}_3
\end{pmatrix}
=
\begin{pmatrix}
	x_0 + i\, c\, t_0\\
	x_1 + i\, c\, t_1\\
	x_2 + i\, c\, t_2\\
	x_3 + i\, c\, t_3
\end{pmatrix}
\end{equation}

If we have to give the event a name, we exchange x with the name or (if it's importaant to indicate the nature of the value) add the name in scriptstyle:

\begin{equation}
\underline{x{\scriptstyle a}}_i
=
\begin{pmatrix}
	\underline{x{\scriptstyle a}}_0\\
	\underline{x{\scriptstyle a}}_1\\
	\underline{x{\scriptstyle a}}_2\\
	\underline{x{\scriptstyle a}}_3
\end{pmatrix}
=
\begin{pmatrix}
	x{\scriptstyle a}_0 + i\, c\, t{\scriptstyle a}_0\\
	x{\scriptstyle a}_1 + i\, c\, t{\scriptstyle a}_1\\
	x{\scriptstyle a}_2 + i\, c\, t{\scriptstyle a}_2\\
	x{\scriptstyle a}_3 + i\, c\, t{\scriptstyle a}_3
\end{pmatrix}
\end{equation}

\subsection{Distance between Event \textit{a} and Event \textit{b}}
\begin{equation}
\underline{\Delta x{\scriptstyle a,b}}_i
=
\begin{pmatrix}
	\underline{{\scriptstyle \Delta}x{\scriptstyle a,b}}_0\\
	\underline{{\scriptstyle \Delta}x{\scriptstyle a,b}}_1\\
	\underline{{\scriptstyle \Delta}x{\scriptstyle a,b}}_2\\
	\underline{{\scriptstyle \Delta}x{\scriptstyle a,b}}_3
\end{pmatrix}
=
\begin{pmatrix}
	{\scriptstyle \Delta}x{\scriptstyle a,b}_0 + i\ c\ {\scriptstyle \Delta}t{\scriptstyle a,b}_0\\
	{\scriptstyle \Delta}x{\scriptstyle a,b}_1 + i\ c\ {\scriptstyle \Delta}t{\scriptstyle a,b}_1\\
	{\scriptstyle \Delta}x{\scriptstyle a,b}_2 + i\ c\ {\scriptstyle \Delta}t{\scriptstyle a,b}_2\\
	{\scriptstyle \Delta}x{\scriptstyle a,b}_3 + i\ c\ {\scriptstyle \Delta}t{\scriptstyle a,b}_3
\end{pmatrix}
=
\begin{pmatrix}
	(x{\scriptstyle b}_0-x{\scriptstyle a}_0) + i\ c\ (t{\scriptstyle b}_0-t{\scriptstyle a}_0)\\
	(x{\scriptstyle b}_1-x{\scriptstyle a}_1) + i\ c\ (t{\scriptstyle b}_1-t{\scriptstyle a}_1)\\
	(x{\scriptstyle b}_2-x{\scriptstyle a}_2) + i\ c\ (t{\scriptstyle b}_2-t{\scriptstyle a}_2)\\
	(x{\scriptstyle b}_3-x{\scriptstyle a}_3) + i\ c\ (t{\scriptstyle b}_3-t{\scriptstyle a})_3
\end{pmatrix}
\end{equation}
This value is oriented (has a sign ...) and dependent on the observer (coordinate system).

\subsection{Squared distance between two Events}
\begin{equation}
\begin{split}
(\underline{\Delta x}_i)^2
&=
\underline{\Delta x}_i \underline{\Delta x}_i
=
(\underline{\Delta x}_0)^2 +
(\underline{\Delta x}_1)^2 +
(\underline{\Delta x}_2)^2 +
(\underline{\Delta x}_3)^2
\\% split here
&=
(\Delta x_0^2+\Delta x_1^2+\Delta x_2^2+\Delta x_3^2)
- c^2\,(\Delta t_0^2+\Delta t_1^2+\Delta t_2^2+\Delta t_3^2)
\\% split here
&\quad + i\, 2\, c\, (\Delta x_0 \Delta t_0 + \Delta x_1 \Delta t_1 +\Delta x_2 \Delta t_2 +\Delta x_3 \Delta t_3)
\end{split}
\end{equation}
This value is not oriented and independent of the observer.

\subsection{Velocity}
The quotients of distance between two closely neighbored events are the components of the velocity tensor:
\begin{equation}
	\underline{v}_{ij}
	=
	\frac{\underline{\Delta x}_i}{\underline{\Delta x}_j}
	=
	\frac{x_i + i\, c\, t_i}{x_j + i\, c\, t_j}
	=
	\begin{pmatrix}
	v_{00} & v_{01} & v_{02} & v_{03}\\
	v_{10} & v_{11} & v_{12} & v_{13}\\
	v_{20} & v_{21} & v_{22} & v_{23}\\
	v_{30} & v_{31} & v_{32} & v_{33}\\
	\end{pmatrix}
\end{equation}
As the distances are observer-dependent this tensor is dependent on the observer as well.




\end{document}
